\documentclass[a4paper,12pt]{article}
\usepackage{quantum-chemistry-bonn}  % here your package is loaded

\title{Example for \texttt{quantum-chemistry-bonn}}
\author{Christian Selzer \& Lukas Wittmann}
\date{\today}

\begin{document}
\maketitle

\section{Introduction}

    In this document, it is demonstrated how to use abbreviations and colors from \texttt{quantum-chemistry-bonn}.

\subsection{Abbreviations}
    Here are a few examples:
    \begin{itemize}
      \item Calculations were performed with \orca\
      \item The best program is \xtb\ for SQM and it has \emph{no} bugs.
      \item The energy is 5 \kcalpmol. Another energy is 4 \kjpmol.
      \item Energies were calculated on the \method{r2scan3c}, or just use \method{rsc} level of theory. Sadly, no-one wants to use \method{wpr2scan50d4}.
    \end{itemize}

\subsection{Colored Elements}
    \begin{itemize}
        \item Here something is written in \textcolor{bonnblue}{blue}, optionally also via \verb|\colb{}|.
        \item For comparison something in \textcolor{blue}{normal blue}.
    \end{itemize}


\section{Conclusion}
    The package offers simple abbreviations and colors that are maintained centrally in one file.

\end{document}
