\documentclass[a4paper,12pt]{article}
%\usepackage{qcsytle}  % hier wird dein Paket geladen
% ------------------------------------------------------------
% qcsytle.sty  v0.1  2025/06/03  
%Package for unifying the writing of common qc program names
% Copyright (c) 2025 Christian Selzer
% ------------------------------------------------------------

%%Identification
%%The package identifies itself and the LaTeX version needed
\NeedsTeXFormat{LaTeX2e}
\ProvidesPackage{qcsytle}[2025/06/03 qcstyle]

\RequirePackage{xcolor}

%==========Colorors========
\definecolor{bonnblue}{RGB}{7, 78, 159}
\definecolor{bonnred}{RGB}{185, 39, 39}
\definecolor{bonnyellow}{RGB}{252, 186, 0}
\definecolor{bonngrey}{RGB}{144, 144, 133}

\definecolor{newaccent}{RGB}{0, 0, 0}
\definecolor{black}{RGB}{0, 0, 0}
\definecolor{highlightgreen}{RGB}{0, 204, 0}
\definecolor{white}{RGB}{255, 255, 255}
\definecolor{StdBody}{RGB}{233,233,233}
\definecolor{bonngreen}{RGB}{0, 123, 78

%==========Programms=======
\newcommand*{\orca}{{\fontfamily{pag}\selectfont ORCA}}
\newcommand*{\draco}{{{\textsc{Draco}}}}
\newcommand*{\crest}{{{\textsc{CREST~}}}}
\newcommand{\xtb}{\texttt{xTB}}

%============Misc==========
\newcommand*{\etal}{\textit{et al.}}

%=========QC-Methods========
\newcommand*{\rsc}{r\textsuperscript{2}SCAN-3c}


\endinput
\title{Beispiel für meinpaket}
\author{Dein Name}
\date{\today}

\begin{document}
\maketitle

\section{Einführung}

    In diesem Dokument demonstriere ich, wie man Abkürzungen und Farben aus \texttt{meinpaket} verwendet.

\subsection{Abkürzungen}
    Hier ein paar Beispiele:
    \begin{itemize}
      \item Rechungen wurden mit \orca~durchgeführt
      \item Das beste Program ist \xtb~für sqm und es hat keine bugs.
    \end{itemize}

\subsection{Farbige Elemente}
    \begin{itemize}
        \item hier ist ewtas in \textcolor{bonnblue}{blau geschrieben}
        \item zum vergleich etwas in \textcolor{blue}{normalen blau}
    \end{itemize}

\section{Fazit}
    Das Paket bietet einfache Abkürzungen und Farben, die zentral in einer Datei gepflegt werden.

\end{document}